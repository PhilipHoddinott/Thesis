%%%%%%%%%%%%%%%%%%%%%%%%%%%%%%%%%%%%%%%%%%%%%%%%%%%%%%%%%%%%
% Paul McKee
% Rensselaer Polytechnic Institute
% 1/31/18
% Master's Thesis
% with Dr. Kurt Anderson
% LaTeX Template: Project Titlepage Modified (v 0.1) by rcx
%%%%%%%%%%%%%%%%%%%%%%%%%%%%%%%%%%%%%%%%%%%%%%%%%%%%%%%%%%%%

\documentclass[12pt]{report}
\usepackage[a4paper]{geometry}
\usepackage[utf8]{inputenc}
\usepackage[myheadings]{fullpage}
\usepackage{fancyhdr}
\usepackage{lastpage}
\usepackage{graphicx, wrapfig, subcaption, setspace, booktabs}
\usepackage[T1]{fontenc}
\usepackage[font=small, labelfont=bf]{caption}
\usepackage{fourier}
\usepackage[protrusion=true, expansion=true]{microtype}
\usepackage[english]{babel}
\usepackage{sectsty}
\usepackage{url, lipsum}
\usepackage{makecell}
\usepackage{amsmath}
\usepackage{setspace}
\usepackage{amsmath}
\usepackage{titlesec}
\usepackage[table,xcdraw]{xcolor}
\titleformat{\section}{\normalfont\fontsize{25}{25}\bfseries}{\thesection}{1em}{}

\usepackage[format=hang,font={small,bf},labelfont=bf]{caption}

\usepackage{listings}
\usepackage{color} %red, green, blue, yellow, cyan, magenta, black, white
\definecolor{mygreen}{RGB}{28,172,0} % color values Red, Green, Blue
\definecolor{mylilas}{RGB}{170,55,241}

\newcommand{\HRule}[1]{\rule{\linewidth}{#1}}
\setcounter{tocdepth}{5}
\setcounter{secnumdepth}{5}

\pagestyle{fancy}
%\fancyhf{Philip Hoddinott, Module 4 Questions}
\fancyhf{}
\fancyhead[L]{Philip Hoddinott}
\fancyhead[C]{Project 2}
\fancyhead[R]{\leftmark}


% %---------------------------------------------------------------
% % HEADER & FOOTER
% %---------------------------------------------------------------

\fancyhf{}
\pagestyle{fancy}
\renewcommand{\headrulewidth}{0pt}
%\setlength\headheight{0pt}
%\fancyhead[L]{ Paul McKee }
%\fancyhead[R]{Rensselaer Polytechnic Institute}
\cfoot{ \thepage\ } 


%--------------------------------------------------------------
% TITLE PAGE
%--------------------------------------------------------------

\begin{titlepage}
	\title{ 
		\LARGE \textbf{\uppercase{Tracking of space debris from publicly available data\\ Think of a better title}} \\
		\vspace{0.25cm}
		\LARGE \textbf{Philip Hoddinott}
	}
	\author{\small{Submitted in Partial Fulfillment of the Requirements} \\ \small{for the Degree of} \\
		\uppercase{Master of Science} \\ \\
		Approved by: People\\
		%\\ Kurt Anderson, Chair \\ John Christian \\ Sandipan Mishra \\ \\ %% from paul's template
		\includegraphics[width=2.5cm]{rensselaer_seal.png} \\
		\small{\textit{Department of Mechanical, Aerospace, and Nuclear Engineering}} \\
		\small{Rensselaer Polytechnic Institute} \\ 
		\small{Troy, New York} \\
		\small{August 2018}
	}
\end{titlepage}

\begin{document}
	\maketitle
	
	\pagenumbering{roman}
	\setcounter{page}{2}
	% --Table of Contents----------
	\tableofcontents
	%\addcontentsline{toc}{section}{\uppercase{Table of Contents}}
	\listoftables
	%\addcontentsline{toc}{section}{\uppercase{List of Tables}}
	\listoffigures
	%\addcontentsline{toc}{section}{\uppercase{List of Figures}}
	% -----------------------------
	
	% ------------------------------------------------------------
	% Acknowledgement
	% ------------------------------------------------------------
	\newpage
	\section{Acknowledgments}
	%\addcontentsline{toc}{section}{\uppercase{Acknowledgement}}
	
	
	Thank lots of people here
	

	
	% ------------------------------------------------------------
	% Abstract 
	% ------------------------------------------------------------
	
	\newpage
	\section{Abstract}

	Talk about Goals of project
	% ------------------------------------------------------------
	% Introduction
	% ------------------------------------------------------------
	\newpage
	\section{Introduction}
	
	Talk more about project. one or two paragraph here 

	\subsection{Space Debris}
	Space Debris is bad
	\subsection{CubeSats}
	talk about cube sats here
	\pagenumbering{arabic} % this should start the normal numberinbg
	
	\subsection{OSCAR}
	Unsure if talking about oscar? Yes not
	\subsection{NORAD /Space Track}
	
	% ------------------------------------------------------------
	% Spacecraft Dynamics
	% ------------------------------------------------------------
	\newpage
	%\section{Data Types}
	\section{Two Line Elements}
	A Two Line Element (TLE) is a data format that encodes a list of orbital elements for an Earth-orbiting object for a given point in time [Re do this]

	
	Stuff about it\par
	An example is given below. The line under the dashes is the reference number line.
	\begin{verbatim}
	ISS (ZARYA)
	1 25544U 98067A   04236.56031392  .00020137  00000-0  16538-3 0  9993
	2 25544  51.6335 344.7760 0007976 126.2523 325.9359 15.70406856328906
	----------------------------------------------------------------------
	1234567890123456789012345678901234567890123456789012345678901234567890   
	1         2         3         4         5         6         7
	
	\end{verbatim}
	Table \ref{tab:TLE_Desc}\cite{SpaceTrackTLE} describes the example TLE. 
	% Please add the following required packages to your document preamble:
	% \usepackage{graphicx}
	% \usepackage[table,xcdraw]{xcolor}
	% If you use beamer only pass "xcolor=table" option, i.e. \documentclass[xcolor=table]{beamer}
	
	% Please add the following required packages to your document preamble:
	% \usepackage{graphicx}
	% \usepackage[table,xcdraw]{xcolor}
	% If you use beamer only pass "xcolor=table" option, i.e. \documentclass[xcolor=table]{beamer}
		\begin{table}[h!]
		\centering
		\caption{Description of TLE}
		\label{tab:TLE_Desc}
		\resizebox{\textwidth}{!}{%
			\begin{tabular}{|l|l|l|}
				\hline
				\multicolumn{3}{|l|}{\textbf{Line 0}}                                                                                                                               \\ \hline
				\rowcolor[HTML]{333333} 
				{\color[HTML]{FFFFFF} \textbf{Columns}} & {\color[HTML]{FFFFFF} \textbf{Example}} & {\color[HTML]{FFFFFF} \textbf{Description}}                                     \\ \hline
				1-24                                    & ISS (ZARYA)                             & The common name for the object based on information from the Satellite Catalog. \\ \hline
				\multicolumn{3}{|l|}{\textbf{Line 1}}                                                                                                                               \\ \hline
				\rowcolor[HTML]{333333} 
				{\color[HTML]{FFFFFF} \textbf{Columns}} & {\color[HTML]{FFFFFF} \textbf{Example}} & {\color[HTML]{FFFFFF} \textbf{Description}}                                     \\ \hline
				1                                       & 1                                       & Line Number                                                                     \\ \hline
				3-7                                     & 25544                                   & Satellite Catalog Number                                                        \\ \hline
				8                                       & U                                       & Elset Classification                                                            \\ \hline
				10-11                                   & 98                                      & International Designator (Last two digits of launch year)                       \\ \hline
				12-14                                   & 067                                     & International Designator (Launch number of the year)                            \\ \hline
				15-17                                   & A                                       & International Designator (Piece of the launch)                                  \\ \hline
				19-32                                   & 04                                      & Epoch Year (last two digits of year)                                            \\ \hline
				21-32                                   & 236.56031392                            & Epoch (day of the year and fractional portion of the day)                       \\ \hline
				34-43                                   & .00020137                               & 1st Derivative of the Mean Motion with respect to Time                          \\ \hline
				45-52                                   & 00000-0                                 & 2nd Derivative of the Mean Motion with respect to Time (decimal point assumed)  \\ \hline
				54-61                                   & 16538-3                                 & B* Drag Term                                                                    \\ \hline
				63                                      & 0                                       & Element Set Type                                                                \\ \hline
				65-68                                   & 999                                     & Element Number                                                                  \\ \hline
				69                                      & 3                                       & Checksum                                                                        \\ \hline
				\multicolumn{3}{|l|}{\textbf{Line 2}}                                                                                                                               \\ \hline
				\rowcolor[HTML]{333333} 
				{\color[HTML]{FFFFFF} \textbf{Columns}} & {\color[HTML]{FFFFFF} \textbf{Example}} & {\color[HTML]{FFFFFF} \textbf{Description}}                                     \\ \hline
				1                                       & 2                                       & Line Number                                                                     \\ \hline
				3-7                                     & 25544                                   & Satellite Catalog Number                                                        \\ \hline
				9-16                                    & 51.6335                                 & Orbit Inclination (degrees)                                                     \\ \hline
				18-25                                   & 344.7760                                & Right Ascension of Ascending Node (degrees)                                     \\ \hline
				27-33                                   & 0007976                                 & Eccentricity (decimal point assumed)                                            \\ \hline
				35-42                                   & 126.2523                                & Argument of Perigee (degrees)                                                   \\ \hline
				44-51                                   & 325.9359                                & Mean Anomaly (degrees)                                                          \\ \hline
				53-63                                   & 15.70406856                             & Mean Motion (revolutions/day)                                                   \\ \hline
				64-68                                   & 32890                                   & Revolution Number at Epoch                                                      \\ \hline
				69                                      & 6                                       & Checksum                                                                        \\ \hline
			\end{tabular}%
		}
	\end{table}
	
	\textcolor{red}{to do here, add more details on what the checksum and stuff liek that is}
	\subsection{SatCat?}
	\section{NORAD Space-Track}	
	Describe the site
	desribe the SATCAT, then the TLE query
	
	
	\subsection{Space-Track Query}
	\subsection{matlab code?}
	
	
	\section{Conclusion}
		
		%--------------------------------------
		% References
		% -------------------------------------
		
		\bibliographystyle{unsrt}
		\bibliography{ref}
		\begin{thebibliography}{10}
			\addcontentsline{toc}{section}{Bibliography}
			
			\bibitem{Hold} %1
			Hold for now before I used the bib file
			
			
		\end{thebibliography}
		
		%-----------------------------------------------------------
		% Appendix
		%-----------------------------------------------------------
		\newpage
		\section*{Appendix 1 - MATLAB code}
		\addcontentsline{toc}{section}{Appendix}
		
		\lstset{language=Matlab,%
			%basicstyle=\color{red},
			breaklines=true,%
			morekeywords={matlab2tikz},
			keywordstyle=\color{blue},%
			morekeywords=[2]{1}, keywordstyle=[2]{\color{black}},
			identifierstyle=\color{black},%
			stringstyle=\color{mylilas},
			commentstyle=\color{mygreen},%
			showstringspaces=false,%without this there will be a symbol in the places where there is a space
			numbers=left,%
			numberstyle={\tiny \color{black}},% size of the numbers
			numbersep=9pt, % this defines how far the numbers are from the text
			emph=[1]{for,end,break},emphstyle=[1]\color{red}, %some words to emphasise
			%emph=[2]{word1,word2}, emphstyle=[2]{style},    
		}
		
	Thanks for Paul McKee who started this template. It seems to have good matlab code viwing
		
	\end{document}
	
